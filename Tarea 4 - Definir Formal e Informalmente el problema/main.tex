\documentclass[11pt,letterpaper]{article}
\usepackage[activeacute,spanish]{babel}
%\usepackage[ansinew]{inputenc}
\usepackage[utf8]{inputenc}
% \usepackage[latin1]{inputenc}
\usepackage[letterpaper,includeheadfoot, top=0.5cm, bottom=3.0cm, right=2.0cm, left=2.0cm]{geometry}
\renewcommand{\familydefault}{\sfdefault}

\usepackage{graphicx}
\usepackage{color}
\usepackage{hyperref}
\usepackage{amssymb}
\usepackage{url}
%\usepackage{pdfpages}
\usepackage{fancyhdr}
\usepackage{hyperref}
\usepackage{subfig}
\usepackage{colortbl}

\usepackage{listings} %Codigo
\lstset{language=C, tabsize=4,framexleftmargin=5mm,breaklines=true}

\begin{document}
%\begin{sf}
% --------------- ---------PORTADA --------------------------------------------
\newpage
\pagestyle{fancy}
\fancyhf{}
%------------------ TÍTULO -----------------------
\vspace*{6cm}
\begin{center}
\Huge  {Definiendo Formal e Informalmente el Problema Planteado}
\vspace{1cm}
\end{center}
%----------------- NOMBRES ------------------------
\begin{center}
\begin{tabular}{ll}
Autor: & José Luis Pérez Avila\\
Profesor & Said Polanco Maragon\\
& Cd. Victoria, Tamaulipas.
\end{tabular}
\end{center}
%\date{}

\section{Definicion Informal del Problema}
Encontar un modelo de programación en el lenguaje de python que sea posible utilizar en una computadora embebida y que nos ayude a la identificación de rostros del piloto y copiloto con una presición de 95\% o más por medio de fotos tomadas en el habitáculo por una camara de un vehículo de gamma media baja para que alerte al dueño del vehículo y a una segunda persona cuando un desconocido ingrese al mismo por medio de una aplicación móvil.

\section{Definiendo Formalmente el Problema}
Un problema bien definio que tiene propiedades analíticas adecuadas y cuyas soluciones posibles tienen una estructura conveniente. Suelen incluir:

\begin{itemize}
\item[1]{La Existencia de alguna solución.}
\item[2]{La unicidad de la solución.}
\item[3]{La solución depende de manera continua de las condiciones iniciales.}
\end{itemize}

Un problema del tipo planteado en el espacio de Banach está bien propuesto en el sentido de Hadarmad si tiene las tres propiedades siguientes:
\begin{itemize}
\item[1]{\textbf{Unicidad}:Las soluciones estrictas están determinadas unívocamente por las condiciones iniciales.}
\item[2]{\textbf{Conjunto denso}: El conjunto \textit{u} de todas las condiciones iniciales correspondientes a las soluciones posibles es denso en el espacio de Banach en el que se plantea el problema.}
\item[3]{\textbf{Acotación local}: Para algún intervalo finito $[0,t_0]$ existe una constante $K = K(t_0)$ tal que cada solución estricta satisface la desigualdad: $\left \| u(t) \right \| \leq K \left \| u_0 \right \|$ para $0 \leq t \leq t_0$ }
\end{itemize}
\end{document}













